\chapter{Experiments and Results}


\section{Data}


\begin{table}[htbp]
  \centering
  \caption[GAP Column]{GAP Dataset Overview}
  \label{tab:template-files}
    \begin{tabularx}{\linewidth}{clX}
      \toprule[1.5pt]
      \textbf{Column} & \textbf{Header} & \textbf{Description} \\\midrule[1pt]
      1 & ID & Unique identifer for an example (two pairs)\\
      2 & Text & Text containing the ambiguous pronoun and two candidate names. About a paragraph in length\\
      3 & Pronoun & The pronoun, text\\
      4 & Pronoun-offset	 & Character offset of Pronoun in Column 2 (Text)\\
      5 & A & The first name, text\\
      6 & A-offset	 & Character offset of A in Column 2 (Text)\\
      7 & A-coref	 & Whether A corefers with the pronoun, TRUE or FALSE\\
      8 & B & The second name, text\\
      9 & B-offset	 & Character offset of B in Column 2 (Text)\\
      10 & B-coref	 & Whether B corefers with the pronoun, TRUE or FALSE\\
      11 & URL & The URL of the source Wikipedia page\\
      \bottomrule[1.5pt]
    \end{tabularx}
\end{table}

Following Kaggle competition “Gendered Pronoun Resolution”, 4 for each abstract from Wikipedia pages we are given a pronoun, and we try to predict the right coreference for it, i.e. to which named entity (A or B) it refers. Let’s take a look at this simple example:
“John entered the room and saw [A] Julia. [Pronoun] She was talking to [B] Mary Hendriks and looked so extremely gorgeous that John was stunned and couldn’t say a word.”
Here “Julia” is marked as entity A, “Mary Hendriks” – as entity B, and pronoun “She” is marked as Pronoun. In this particular case the task is to correctly identify to which entity the given pronoun refers.

\section{Extract features with BERT}

\section{Fine-tune BERT}


